\documentclass[aspectratio=169]{beamer}
\usetheme{Boadilla}

\usepackage{mathtools}
\usepackage{graphicx}
\usepackage{tikz}

\graphicspath{ {./images/} }
\setbeamertemplate{navigation symbols}{}

\newcommand{\cntxt}{\ \mathrm{context}}
\newcommand{\typ}{\ \mathrm{type}}
\newcommand{\N}{\mathbb{N}}
\newcommand{\app}[2]{App_{[x:\sigma]\tau}(#1, #2)}
\newcommand{\pair}[2]{Pair_{[x:\sigma]\tau}(#1, #2)}
\newcommand{\R}[2]{\mathrm{R}_{[z:\Sigma x:\sigma.\tau]\rho}^{\Sigma}(#1, #2)}
\newcommand{\RN}[3]{\mathrm{R}_{[n: \N]\sigma}^{\N}(#1, #2, #3)}
\newcommand{\C}{Comp}
\newcommand{\Intro}{Intro}
\newcommand{\F}{Form}
\newcommand{\E}{Elim}
\newcommand\defeq{\stackrel{\mathclap{\normalfont\mbox{def}}}{\ =\ }}
\newcommand{\Id}[2]{Id_\sigma(#1,#2)}
\newcommand{\Refl}[1]{Refl_\sigma(#1)}
\newcommand{\RID}[4]{\mathrm{R}_{[x:\sigma,y:\sigma,p:Id_\sigma(x,y)]\tau}^{Id}(#1, #2, #3, #4)}
\newcommand{\bn}{[n:\N,x:\sigma]}
\newcommand{\bi}{[z:\sigma]}
\newcommand{\Pich}{\hat{\Pi}}
\newcommand{\Appch}[2]{\hat{App}_{[x:\sigma]El(T)}(#1,#2)}
\newcommand{\lambdach}{\hat{\lambda} x:\sigma.M^{El(T)}}
\newcommand{\Gamdash}{\Gamma\vdash}
\newcommand{\Fami}{\mathcal{F}am}
\newcommand{\cate}{\mathcal{C}}
\newcommand{\types}{Ty_{\cate}}
\newcommand{\terms}{Tm_{\cate}}

\title{Dependant Types}
\subtitle{Initiation to research}
\author{Louis Milhaud}
\institute{Université Paris Saclay}
\date{\today}

\begin{document}
    \begin{frame}
        \titlepage
    \end{frame}
    \begin{frame}
        \frametitle{Outline}
        \tableofcontents
    \end{frame}
    \section{Introduction}
    \subsection{History}
    \begin{frame}
        \frametitle{History}
        \begin{itemize}
            \item[-] 1972: System F, J-Y. Girard
            \item[-] 1975: \underline{Intuitionnistic Type Theory}, P. Martin-Lof
            \item[-] 1986: Nuprl System
            \item[-] 1988: Calculus of Construction, T. Coquand  
            \item[-] 1989: Coq
            \item[-] 1997: \underline{Syntax \& Semantics of Dependant Type Theory}, M. Hofmann 
        \end{itemize}
    \end{frame}
    \subsection{Definition}
    \begin{frame}
        \frametitle{Definition}
        A dependant type is a family of types varying on the elements of another type.\\
        \vspace{20pt}
        \underline{Example:}\\
        $$Vec_\sigma(M),\quad M:\N $$
        with the following objects:
        \begin{itemize}
            \item $nil_\sigma : Vec_\sigma(0)$
            \item $Cons_\sigma(U, V) : Vec_\sigma(Succ(M))$ 
        \end{itemize}
        with $ U : \sigma\ \text{ and } V : Vec_\sigma(M)$
    \end{frame}
    \subsection{Judgments \& Equality}
    \begin{frame}
        \frametitle{Judgments}
        \begin{align*}
            &\vdash \Gamma \cntxt &\Gamma \text{ is a valid context}\\
            \Gamma &\vdash \sigma \typ &\sigma \text{ is a type in context $\Gamma$}\\
            \Gamma &\vdash M : \sigma &M\text{ is a term of type $\sigma$ in context $\Gamma$}\\
            &\vdash \Gamma = \Delta \cntxt &\text{$\Gamma$ and $\Delta$ are definitionaly equal contexts}\\
            \Gamma &\vdash \sigma = \tau &\text{$\sigma$ and $\tau$ are def. equal types in context $\Gamma$}\\
            \Gamma &\vdash M = N : \sigma &M\text{ and }N\text{ are def. equal terms of $\sigma$ in $\Gamma$}
        \end{align*}
    \end{frame}
    \begin{frame}
        \frametitle{Definitional Equality}
        \underline{Definitional Equality (Per Martin-Lof):}\\
        \vspace{10pt}
        \emph{"Definitional equality is intensional equality, or equality of meaning."}\\
        \vspace{10pt}
        This equality defines an equivalence relation:
        $$\frac{\vdash \Gamma \cntxt}{\vdash \Gamma = \Gamma \cntxt}\qquad \frac{\vdash \Gamma = \Delta \cntxt}{\vdash \Delta = \Gamma \cntxt}$$
        $$\frac{\vdash \Gamma = \Theta \cntxt \vdash \Theta = \Delta \cntxt}{\vdash \Gamma = \Delta \cntxt}$$
        With the same rules for types and terms.
    \end{frame}
    \begin{frame}
        \frametitle{Notions of Equality}
        \underline{Three notions of equality:}
        \begin{itemize}
            \item[-] Judgmental equality
            $$\frac{\Gamma \vdash \N \typ}{\Gamma \vdash 1:\N} \qquad \frac{\Gamma \vdash \N \typ}{\Gamma \vdash 1 = Suc(0):\N}$$
            \item[-] Typal equality
            $$\frac{\Gamma \vdash \N \typ}{\Gamma \vdash 1:\N} \qquad \frac{\Gamma \vdash \N \typ}{\Gamma \vdash \delta_1:1 =_{\N} Suc(0)}$$
            \item[-] Propositional equality
            $$\frac{\Gamma \vdash \N \typ}{\Gamma \vdash 1:\N} \qquad \frac{\Gamma \vdash \N \typ}{\Gamma \vdash 1 =_{\N} Suc(0) \; \mathrm{true}}$$
        \end{itemize}
    \end{frame}
    \subsection{Basic rules}
    \begin{frame}
        \frametitle{Basic rules 1}
        \underline{Rules for context formation:}
        $$\frac{}{\vdash \diamond \cntxt} \qquad \frac{\Gamma \vdash \sigma \typ}{\vdash \Gamma, x: \sigma \cntxt}$$
        $$\frac{\Gamma = \Delta \cntxt \quad \Gamma \vdash \sigma = \tau \typ}{\vdash \Gamma, x: \sigma = \Delta, y:\tau \cntxt}$$
        \underline{Variable Rule:}
        $$\frac{\vdash \Gamma, x: \sigma, \Delta \cntxt}{\Gamma, x:\sigma,\Delta\vdash x : \sigma}$$
    \end{frame}
    \begin{frame}
        \frametitle{Basic rules 2}
        \underline{Rules for typing and definitional equality:}
        $$\frac{\Gamma \vdash M : \sigma \quad \vdash \Gamma = \Delta \cntxt \quad \Gamma \vdash \sigma = \tau \typ}{\Delta \vdash M : \tau}$$
        $$\frac{\vdash \Gamma = \Delta \cntxt \quad \Gamma \vdash \sigma \typ}{\Delta \vdash \sigma \typ}$$
        \underline{Weakening and substitution Rules:}
        $$\frac{\Gamma, \Delta \vdash \mathcal{J} \quad \Gamma \vdash \rho \typ}{\Gamma, x: \rho, \Delta\vdash \mathcal{J}}$$
        $$\frac{\Gamma,x:\rho,\Delta \vdash \mathcal{J}\quad \Gamma \vdash U: \rho}{\Gamma,\Delta[U/x]\vdash \mathcal{J}[U/x]}$$
    \end{frame}
    \section{Type formers}
    \subsection{$\prod$-type}
    \begin{frame}
        \frametitle{$\prod$-type}
        \begin{itemize}
            \item Type former for the functions with return type depending on the parameter
            \item \underline{Set-theoretic equivalent:}\\
            Cartesian product over a family of sets: $\Pi_{i\in I}B_i$
            \item preserved by definitional equality.
        \end{itemize}
    \end{frame}
    \begin{frame}
        \frametitle{Rules}
        $$\frac{\Gamma \vdash \sigma \typ \quad \Gamma,x:\sigma\vdash \tau \typ}{\Gamma \vdash \Pi x:\sigma.\tau \typ}\F$$
        \vspace{10pt}
        $$\frac{\Gamma, x:\sigma \vdash M : \tau}{\Gamma \vdash \lambda x:\sigma.M^\tau : \Pi x:\sigma.\tau}\Intro$$
        \vspace{10pt}
        $$\frac{\Gamma \vdash \lambda x:\sigma.M^\tau:\Pi x:\sigma.\tau \quad \Gamma \vdash N:\sigma}{\Gamma \vdash \app{\lambda x:\sigma.M^{\tau}}{N} = M[N/x]:\tau[N/x]}\C$$
        \vspace{10pt}
        $$\frac{\Gamma \vdash M : \Pi x: \sigma . \tau \quad \Gamma \vdash N: \sigma}{\Gamma \vdash \app{M}{N}: \tau[N/x]}\E$$
    \end{frame}
    \subsection{$\sum$-type}
    \begin{frame}
        \frametitle{$\sum$-type}
        \begin{itemize}
            \item Type former for pairs with second element type depending on the first element
            \item \underline{Set-theoretic equivalent:}\\
            Disjoint union over a family of sets: $\Sigma_{i\in I}B_i \defeq \{(i,b)|i \in I \wedge b \in B_i\}$
            \item $\pi_1 \defeq \R{[x:\sigma,y:\tau]x}{M}:\sigma$
            \item $\pi_2 \defeq \mathrm{R}^\Sigma_{[z:\Sigma x:\sigma.\tau]\tau[z.1/x]}([x:\sigma,y:\tau]y, M):\tau[M.1]$
            \item preserved by definitional equality.
        \end{itemize}
    \end{frame}
    \begin{frame}
        \frametitle{Rules}
        $$\frac{\Gamma \vdash \sigma \typ \quad \Gamma,x:\sigma\vdash \tau \typ}{\Gamdash \Sigma x:\sigma.\tau \typ}\F$$
        $$\frac{\Gamma \vdash M : \sigma\quad \Gamma \vdash N : \tau[M/x]}{\Gamma \vdash \pair{M}{N} : \Sigma x:\sigma.\tau}\Intro$$
        $$\frac{\Gamma \vdash \R{[x:\sigma,y:\tau]H}{\pair{M}{N}:\rho[\pair{}{}/z]}}{\Gamma \vdash \R{[x:\sigma,y:\tau]H}{\pair{M}{N}}}\C$$
        $$=H[M/x,N/y]:\rho[\pair{}{}/z].$$
        $$\Gamma,z:\Sigma x:\sigma.\tau \vdash \rho \typ$$
        $$\Gamma,x:\sigma,y:\tau\vdash H:\rho [\pair{x}{y}/z]$$
        $$\frac{\quad \Gamma \vdash M : \Sigma x:\sigma.\tau}{\Gamma \vdash \R{[x:\sigma.y:\tau]H}{M}:\rho[M/z]}\E$$
    \end{frame}
    \subsection{Naturals}
    \begin{frame}
        \frametitle{Naturals}
        Naturals are as always inductively defined, thus we have the two following objects:
        $$0 \text{ and } Suc(M) \text{ with } M:\N$$
        \vspace{10pt}
        Then we can define operations like addition:
        $$M + N \defeq \RN{N}{[n:\N, x:\N]Suc(x)}{M}:\N$$
    \end{frame}
    \begin{frame}
        \frametitle{Rules}
        $$\frac{\vdash \Gamma \cntxt}{\Gamma \vdash \N \typ}\F \frac{\vdash \Gamma \cntxt}{\Gamma \vdash 0 : \N}\& \frac{\Gamma \vdash M : \N}{\Gamma \vdash Suc(M):\N}\Intro$$
        $$\Gamma,n:\N\vdash\sigma\typ$$
        $$\Gamdash H_z : \sigma[0/n]$$
        $$\Gamma,n:\N,x:\sigma\vdash H_s : \sigma[Suc(n)/n]$$
        $$\frac{\Gamdash M: \N}{\Gamdash\RN{H_z}{[n:\N,x:\sigma]H_s}{M}:\sigma[M/n]}\E$$
        $$\frac{\Gamma \vdash \RN{H_z}{\bn H_s}{0} : \sigma[0/n]}{\Gamma \vdash \RN{H_z}{\bn H_s}{0} = H_z: \sigma[0/n]}\C$$
        $$\frac{\Gamma \vdash \RN{H_z}{\bn H_s}{Suc(M)} : \sigma[Suc(M)/n]}{\Gamma \vdash \RN{H_z}{\bn H_s}{Suc(M)} =}\C$$
        $$ H_s[M/n,\RN{H_z}{\bn H_s}{M}/x]:\sigma[Suc(M)/n]$$
    \end{frame}
    \subsection{Identity types}
    \begin{frame}
        \frametitle{Identity types}
        \begin{itemize}
            \item Type former for typal equality
            \item Enables equality reasonning inside type theory
            \item $\Refl{M} : \Id{M}{M}$ with $M : \sigma$
            \item We can deduce other properties of typal equality like symmetry, transitivity, Leibniz principle thanks to the elimination rule.
        \end{itemize}
    \end{frame}
    \begin{frame}
        \frametitle{Rules}
        $$\frac{\Gamdash M:\sigma \quad \Gamdash N:\sigma}{\Gamma \vdash \Id{M}{N} \typ}\F$$
        $$\frac{\Gamdash M:\sigma}{\Gamdash\Refl{M}:\Id{M}{M}}\Intro$$
        $$\Gamdash\sigma\typ\quad\Gamdash M:\sigma\quad\Gamdash N:\sigma$$
        $$\Gamma, x:\sigma,y:\sigma,p:\Id{x}{y}\vdash\tau\typ$$
        $$\Gamma,z:\sigma\vdash H:\tau[z/x,z/y,\Refl{z}/p]$$
        $$\frac{\Gamdash P:\Id{M}{N}}{\Gamdash\RID{[z:\sigma]H}{M}{N}{P}:\tau[M/x,N/y,P/p]}\E$$
        $$\frac{\Gamma \vdash \RID{\bi H}{M}{M}{\Refl{M}}:\tau[M/x,M/y,\Refl{M}/p]}{\Gamma \vdash\ idem = H[M/z]:\tau[M/x,M/y,\Refl{M}/p]}$$$\C$
    \end{frame}
    \subsection{Universes}
    \begin{frame}
        \frametitle{Universes}
        \begin{itemize}
            \item Universes are type formers containing codes for types
            \item We have the two basic rules:
            $$\frac{\vdash \Gamma \cntxt}{\Gamma \vdash U \typ}\quad\frac{\Gamma \vdash M:U}{\Gamdash El(M)\typ}$$
            \item $M$ is a code of the universe $U$
            \item $El(M)$ is the type associated to this code
            \item Add type formers to our Universes
            \item Rules to maintain closure
        \end{itemize}
          
    \end{frame}
    \begin{frame}
        \frametitle{Adding dependant type $\Pich$}
        $$\frac{\Gamma \vdash S:U\quad \Gamma,s:El(S)\vdash T:U}{\Gamma \vdash \Pich(S,[s:El(S)]T):U}\F$$
        \vspace{15pt}
        $$\frac{\Gamdash\Pich(S,[s:El(S)]T):U}{\Gamdash El(\Pich(S,[s:El(S)]T)) = \Pi s:El(S).El(T) \typ}$$
    \end{frame}
    \begin{frame}
        \frametitle{Impredicative quantification}
        \begin{itemize}
            \item We can add $\forall$ or Impredicative quantification
            \item Terms like $\forall x:\sigma.T:U$
            \item domain type $\sigma arbitrarily big$
            \item Allows typing of the identity function $polyone$ known from polymorphic lambda calculus
            \item $polyone = \forall c:U.\forall:El(c).c$
        \end{itemize}
    \end{frame}
    \begin{frame}
        \frametitle{Impredicative quantification - Rules}
        $$\frac{\Gamdash\sigma\typ\quad\Gamma,x:\sigma\vdash T:U}{\Gamdash\forall x:\sigma.T:U}$$
        $$\frac{\Gamma,x:\sigma\vdash M:El(T)}{\Gamdash\lambdach:El(\forall x:\sigma.T)}\Intro$$
        $$\frac{\Gamdash M:el(\forall x:\sigma.T)\quad\Gamdash N:\sigma}{\Gamdash\Appch{M}{N}:El(T[N/x])}\E$$
        $$\frac{\Gamdash\Appch{\lambdach}{N}:El(T[N/x])}{\Gamdash\Appch{\lambdach}{N}=M[N/x]:El(T[N/x])}\C$$
    \end{frame}
    \section{Intensional and Extensional type theory}
    \begin{frame}
        \frametitle{Intensional and Extensional type theory}
        \begin{itemize}
            \item type theory + identity types = intensional type theory
            \item equality reflection:
            $$\frac{\Gamma \vdash M = N : \sigma}{\Gamma \vdash P: \Id{M}{N}}$$
            \item type theory + identity types + reflection rule = extensional type theory
            \item Judgmental equality and typing undecidable
        \end{itemize}
    \end{frame}
    \section{Categoretical Semantics}
    \begin{frame}
        \frametitle{Context morphisms}
        Let $\Gamma$ and $\Delta \defeq x_1:\sigma_1,...,x_n:\sigma_n$ be valid contexts.\\
        $f \defeq (M_1,...,M_n)$ is a context morphism (we write $\Gamdash f \implies \Delta$)\\
        when the following $n$ judgements hold:
        \begin{align*}
            &\Gamdash M_1:\sigma_1\\
            &\Gamdash M_2:\sigma_2[M_1/x_1]\\
            &\dots\\
            &\Gamdash M_n:\sigma_n[M_1/x_1][M_2/x_2]\dots[M_{n-1}/x_{n-1}]
        \end{align*}
    \end{frame}
    \begin{frame}
        \frametitle{Categoretical Semantics}
        Our basic Syntax contains now 4 objects
        \begin{itemize}
            \item[-] contexts
            \item[-] contexts morphisms
            \item[-] types in $Ty$
            \item[-] terms in $Tm$   
        \end{itemize}
        \vspace{10pt}
        And we define the corrisponding Semantic:
        \begin{itemize}
            \item[-] $\cate$ a category of contexts and context morphisms
            \item[-] for $\Gamma\in\cate$ a collection $\types(\Gamma)$ of semantic types
            \item[-] for $\Gamma\in\cate$ and $\sigma\in\types$ a collection $\terms(\Gamma,\sigma)$ of semantic terms
        \end{itemize}
        
    \end{frame}
    \begin{frame}
        \frametitle{Category $\Fami$}
        The category $\Fami$ of families of sets has:
        \begin{itemize}
            \item[-] as objects pairs $A = (A^0,A^1)$
            \item[-] as arrows $f$ between $A$ and $B$ a pair $(f^0,f^1)$
        \end{itemize}
        \vspace{15pt}
        We also define the functor $\mathcal{F}:\cate^{op}\to\Fami$ such that:
        $$\mathcal{F}(\Gamma)=(\types(\Gamma),(\terms(\Gamma,\sigma))_{\sigma\in\types(\Gamma)})$$
    \end{frame}
    \begin{frame}
        \frametitle{Category with families}
        We can now define a category with families with:
        \begin{itemize}
            \item[-] a category $\cate$ with terminal object
            \item[-] a functor $\mathcal{F} = (Ty,Tm):\cate^{op}\to \Fami$
            \item[-] a comprehension for each $\Gamma\in\cate$ and $\sigma\in\types(\Gamma)$  
        \end{itemize}
    \end{frame}
\end{document}